\documentclass{article}
\usepackage[utf8]{inputenc}
\usepackage{amsfonts}
\usepackage{amsmath}
\usepackage[margin=3cm]{geometry}
\usepackage{hyperref}

\setlength{\parindent}{0pt}

\author{Vitaliy Pozdnyakov}
\title{High-Dimensional Statistics}
\date{March 2020}

\begin{document}

\maketitle

Solutions to exercises from the book "High-Dimensional Probability", Roman Vershynin, 2020

\url{https://www.math.uci.edu/~rvershyn/}

\bigskip

Notation:
\begin{itemize}
    \item cc -- coffee cup
    \item $\sum := \sum_{i=1}^N$
    \item $\prod := \prod_{i=1}^N$
\end{itemize}

\section{Exercise 2.1.4 (1 cc)}

Let $g \sim N(0, 1)$. Show that for all $t \geq 1$, we have
$$\mathbb E g^2 \mathbf 1\{g>t\} = t \frac{1}{\sqrt{2 \pi}}e^{-{t^2}/{2}} + \mathbb P\{g>t\} \leq \left(t+\frac{1}{t}\right)\frac{1}{\sqrt{2 \pi}}e^{-{t^2}/{2}}$$

\subsection{Solution}
$$\mathbb E g^2 \mathbf 1\{g>t\} 
= \int_{-\infty}^{+\infty}x^2 \mathbf 1\{x>t\} \frac{1}{\sqrt{2 \pi}}e^{-{x^2}/{2}}dx 
= \frac{1}{\sqrt{2 \pi}} \int_{t}^{+\infty}x^2 e^{-{x^2}/{2}}dx$$

Make the substitution $u = x$ and $dv = xe^{-{x^2}/{2}}dx$, then $du = dx$ and $$v = \int dv = \int xe^{-{x^2}/{2}}dx = \left[z = -\frac{x^2}{2}; dz = -xdx\right] = - \int e^z dz = - e^z = - e^{-{x^2}/{2}}$$

Next, integrate by parts $$\int_a^b udv = uv\Big|_a^b - \int_a^b vdu$$

$$\frac{1}{\sqrt{2 \pi}} \int_{t}^{+\infty}x^2 e^{-{x^2}/{2}}dx 
= \frac{1}{\sqrt{2 \pi}}\left(-xe^{-{x^2}/{2}}\Big|_t^{+\infty}\right) + \int_{t}^{+\infty}\frac{1}{\sqrt{2 \pi}}e^{-{x^2}/{2}} dx 
= t \frac{1}{\sqrt{2 \pi}}e^{-{t^2}/{2}} + \mathbb P\{g>t\}$$

Next, apply the theorem "Tails of the normal distribution":

$$\mathbb P\{g>t\} \leq \frac{1}{t}\cdot \frac{1}{\sqrt{2 \pi}}e^{t^2/2}$$

Thereby,
$$t\frac{1}{\sqrt{2 \pi}}e^{-{t^2}/{2}} + \mathbb P\{g>t\} \leq t\frac{1}{\sqrt{2 \pi}}e^{-{t^2}/{2}} + \frac{1}{t}\cdot \frac{1}{\sqrt{2 \pi}}e^{t^2/2}  = \left(t+\frac{1}{t}\right)\frac{1}{\sqrt{2 \pi}}e^{-{t^2}/{2}}$$

\section{Exercise 2.2.3 (1 cc)}

(Bounding the hyperbolic cosine) 

Show that $$\cosh(x) \leq \exp(x^2/2) \text{ for all } x \in \mathbb R.$$

\subsection{Solution}

Apply Taylor's expansion for the left side of inequation

$$\cosh(x)' = \sinh(x);\
\cosh(x)'' = \cosh(x);\
\cosh(x)''' = \sinh(x);\ 
\dots$$

$$\cosh(0) = 1;\
\sinh(0) = 0$$

Then

$$
\begin{gathered}
    \cosh(x) = \cosh(0) + \frac{\sinh(0)}{1!}x^1 + \frac{\cosh(0)}{2!}x^2 + \frac{\sinh(0)}{3!}x^3 + \dots\\ 
    = 1 + \frac{x^2}{2} + \frac{x^4}{24} + \frac{x^6}{720} + O(x^8)
\end{gathered}$$

For the right side

$$\exp(x^2/2)' = \exp(x^2/2)x;\
\exp(x^2/2)'' = \exp(x^2/2)(x^2+1);\
\exp(x^2/2)''' = \exp(x^2/2)x(x^2+3);\ 
\dots$$

$$
\exp(0^2/2) = 1;\
\exp(0^2/2)0 = 0;\
\exp(0^2/2)(0^2+1) = 1;\
\exp(0^2/2)0(0^2+3) = 0;\ 
\dots$$

Then

$$
\begin{gathered}
    \exp(x^2/2) = \exp(0^2/2) + \frac{\exp(0^2/2)0}{1!}x^1 + \frac{\exp(0^2/2)(0^2+1)}{2!}x^2 + \frac{\exp(0^2/2)0(0^2+3)}{3!}x^3 + \dots\\ 
    = 1 + \frac{x^2}{2} + \frac{x^4}{8} + \frac{x^8}{384} + O(x^8)
\end{gathered}$$

Obviously,

$$1 + \frac{x^2}{2} + \frac{x^4}{24} + \frac{x^6}{720} + O(x^8) \leq 1 + \frac{x^2}{2} + \frac{x^4}{8} + \frac{x^8}{384} + O(x^8)$$

Thus,

$$\cosh(x) \leq \exp(x^2/2) \text{ for all } x \in \mathbb R.$$

\section{Exercise 2.2.7. (2 cc)}

Let $X_1 , . . . , X_N$ be independent random variables. Assume that $X_i \in [m_i , M_i ]$ for every $i$. Prove that for any $t > 0$

$$\mathbb P \left\{ \sum (X_i - \mathbb E X_i) \geq t \right\} \leq \exp\left( -\frac{2t^2}{\sum(M_i - m_i)^2}\right)$$

\subsection{Solution}

Let $\xi_i = X_i - \mathbb E X_i$. Apply Markov's inequality and a property of the exp function

$$\mathbb P \left\{ \sum \xi_i \geq t \right\} = \mathbb P \left\{ e^{\lambda \sum \xi_i} \geq e^{\lambda t} \right\} \leq e^{-\lambda t} \mathbb E \left[ e^{\lambda \sum \xi_i}\right]  \text{ for any } \lambda > 0 $$

Apply independence assumption

$$\mathbb E \left[ e^{\lambda \sum \xi_i}\right] = \mathbb E \left[ \prod e^{\lambda \xi_i}\right] = \prod \mathbb E \left[ e^{\lambda \xi_i}\right]$$

Apply Hoeffding's Lemma

$$\mathbb E \left[ e^{\lambda \xi_i}\right] \leq e^{\frac{\lambda^2(M_i - m_i)^2}{8}}$$

Thus,

$$\mathbb P \left\{ \sum \xi_i \geq t \right\} \leq e^{-\lambda t} \prod e^{\frac{\lambda^2(M_i - m_i)^2}{8}} = e^{-\lambda t + \frac{\lambda^2}{8}\sum (M_i - m_i)^2}$$

Let $\lambda = 4t/\sum(M_i - m_i)^2$, then

$$\mathbb P \left\{ \sum (X_i - \mathbb E X_i) \geq t \right\} \leq \exp\left( -\frac{2t^2}{\sum(M_i - m_i)^2}\right)$$

\section{Exercise 2.2.8. (2 cc)}

(Boosting randomized algorithms)

Imagine we have an algorithm for solving some decision problem (e.g. is a given number $p$ a prime?). Suppose the algorithm makes a decision at random and returns the correct answer with probability $1/2 + \delta$ with some $\delta > 0$, which is just a bit better than a random guess. To improve the performance, we run the algorithm $N$ times and take the majority vote. Show that, for any $\epsilon \in (0, 1)$, the answer is correct with probability at least $1 - \epsilon$, as long as

$$N \leq \frac{1}{2\delta^2}\ln(1/\epsilon)$$

\subsection{Solution}

Let $X_1, X_2, \dots X_N$ be a random sequence with probabilities $$\mathbb P \{X_i = 0\} = 1/2 - \delta$$ $$\mathbb P \{X_i = 1\} = 1/2 + \delta$$

Then the event $\{\text{The answer is incorrect}\}$ can be written as 

\begin{equation*}
    \begin{aligned}
    \left\{ \sum X_i \geq N/2 \right\} = & \left\{ \sum (X_i - \mathbb E X_i) \geq \sum (1/2 - \mathbb E X_i) \right\} \\
    = & \left\{ \sum (X_i - \mathbb E X_i) \geq \sum (1/2 - 1/2 + \delta) \right\} \\
    = & \left\{ \sum (X_i - \mathbb E X_i) \geq \delta N \right\}
    \end{aligned}
\end{equation*}

Apply Hoeffding inequality

$$\mathbb P \left\{ \sum ( X_i -  \mathbb E X_i) \geq t \right\} \leq \exp\left(-\frac{2t^2}{\sum (M_i - m_i)^2}\right)$$

Make substitutions $t = \delta N$, $M_i = 1$, $m_i = 0$

$$\mathbb P \left\{ \sum (X_i - \mathbb E X_i) \geq \delta N \right\} \leq \exp(-2\delta^2 N)$$

We can observe that 
$$\left\{N \leq \frac{1}{2\delta^2}\ln(1/\epsilon)\right\} \Rightarrow \left\{\exp(-2\delta^2 N) < \epsilon\right\}$$
Thereby, $\mathbb P \{\text{The answer is incorrect}\} \leq \epsilon$ and $\mathbb P \{\text{The answer is correct}\} \geq 1 - \epsilon$ under that constraint.

\section{Exercise 2.2.10 (2 cc)}
Let $X_1, \dots , X_N$ be non-negative independent random variables with continuous distributions. Assume that the densities of $X_i$ are uniformly bounded by $1$.

(a) Show that the MGF of $X_i$ satisfies $\mathbb E e^{-tX_i} \leq 1/t$ for all $t>0$.

(b) Deduce that, for any $\epsilon > 0$, we have $\mathbb P\{\sum X_i \leq \epsilon N\} \leq (e\epsilon)^N$.

\subsection{Solution (a)}

Let $\phi_i(x)$ be a PDF of $X_i$

$$\mathbb E \exp(-tX_i) = \int_0^\infty e^{-tx} \phi_i(x) dx$$

Since $\phi_i(x) \leq 1$ for all $i$ and $x$, then
$$\int_0^\infty e^{-tx} \phi_i(x) dx \leq \int_0^\infty e^{-tx} dx = \frac{1}{t} \text{ for all } t>0$$

\subsection{Solution (b)}

Rewrite the probability as

$$\mathbb P \left\{\sum X_i \leq \epsilon N\right\} = \mathbb P \left\{\sum -X_i/\epsilon \geq -N\right\}$$

Apply Markov's inequality and a property of the exp function

$$\mathbb P \left\{\sum -X_i/\epsilon \geq -N\right\} = \mathbb P \left\{ \exp\left(\sum -X_i/\epsilon\right) \geq \exp(-N) \right\} \leq e^N\prod \mathbb E e^{-X_i/\epsilon}$$

Apply the result from the section (a) and then

$$\mathbb P \left\{\sum X_i \leq \epsilon N\right\} \leq e^N \prod \epsilon = (e\epsilon)^N$$

\section{Exercise 2.3.2 (2 cc)}

(Chernoff’s inequality: lower tails)

Modify the proof of Theorem 2.3.1 to obtain the following bound on the lower tail. For any $t < \mu$, we have

$$\mathbb P\{S_N \leq t\} \leq e^{-\mu}\left(\frac{e\mu}{t}\right)^t$$

\subsection{Solution}

Since $t<\mu$, $t>0$ and $S_N>0$ we can rewrite the probability as

$$\mathbb P\{S_N \leq t\} = \mathbb P\{(t/\mu)^{S_N} \geq (t/\mu)^t\}$$

Apply Markov's inequality

$$\mathbb P\{(t/\mu)^{S_N} \geq (t/\mu)^t\} \leq (\mu/t)^t\mathbb E (t/\mu)^{S_N}$$

Consider the expectation

$$\mathbb E (t/\mu)^{S_N} = \prod \mathbb E (t/\mu)^{X_i}$$

$$\mathbb E (t/\mu)^{X_i} = (t/\mu)^1p_i + (t/\mu)^0(1-p_i) = 1 + [(t/\mu) - 1]p_i \leq \exp \left(\left[\frac{t}{\mu}-1\right]p_i\right)$$

$$\prod \mathbb E (t/\mu)^{X_i} \leq \exp \left(\left[\frac{t}{\mu}-1\right]\sum p_i\right) = \exp \left(\left[\frac{t}{\mu}-1\right]\mu \right) = e^te^{-\mu}$$

Thereby

$$\mathbb P\{S_N \leq t\} \leq e^{-\mu}\left(\frac{e\mu}{t}\right)^t$$

\section{Exercise 2.3.3 (2 cc)}
(Poisson tails)

Let $X \sim Pois(\lambda)$. Show that for any $t > \lambda$, we have

$$\mathbb P\{X \geq t\} \leq e^{-\lambda}\left(\frac{e\lambda}{t}\right)^t$$

\subsection{Solution}

Chernoff’s inequality

$$\mathbb P\{S_N \geq t\} \leq e^{-\mu}\left(\frac{e\mu}{t}\right)^t$$

Assume that $N \to \infty$, $\max p_i \to 0$ and $\mathbb E S_N \to \lambda < \infty$. Then, using Poisson limit theorem we have

$$S_N \to Pois(\lambda)$$

Thereby

$$\mathbb P\{X \geq t\} \leq e^{-\lambda}\left(\frac{e\lambda}{t}\right)^t$$

\section{Exercise 2.3.5 (3 cc)}
(Chernoff’s inequality: small deviations)

Show that, in the setting of Theorem 2.3.1, for $\delta \in (0, 1]$ we have

$$\mathbb P \{|S_N-\mu| \geq \delta\mu\} \leq 2e^{-c\mu\delta^2}$$

where $c > 0$ is an absolute constant.

\subsection{Solution}

Consider Chernoff’s inequality with $t>\mu$ (upper tails). Let $t = \mu+\delta\mu$, so

$$\mathbb P\{S_N - \mu \geq \delta\mu\} \leq e^{-\mu}\left(\frac{e}{1+\delta}\right)^{\mu+\delta\mu}$$

In opposite, if $t<\mu$ and $t = \mu-\delta\mu$ (lower tails) we obtain

$$\mathbb P\{S_N - \mu \leq -\delta\mu\} \leq e^{-\mu}\left(\frac{e}{1-\delta}\right)^{\mu-\delta\mu}$$

Apply addition theorem of probability for lower and upper tails

$$
\mathbb P \left( \{S_N - \mu \geq \delta\mu\} \cup \{S_N - \mu \leq -\delta\mu\}\right) = 
\mathbb P \{ |S_N - \mu|  \geq \delta\mu\}$$

$$\mathbb P \{ |S_N - \mu|  \geq \delta\mu\} \leq e^{-\mu}\left(\frac{e}{1-\delta}\right)^{\mu-\delta\mu} + e^{-\mu}\left(\frac{e}{1+\delta}\right)^{\mu+\delta\mu} \leq 2e^{-\mu}\left(\frac{e}{1+\delta}\right)^{\mu+\delta\mu}$$

Consider the inequality

$$
\begin{gathered}
    2e^{-\mu}\left(\frac{e}{1+\delta}\right)^{\mu+\delta\mu} \leq 2e^{-c\mu\delta^2} \\
    \frac{e^{\delta\mu}}{(1+\delta)^{\mu+\delta\mu}} \leq e^{-c\mu\delta^2} \\
    e^{\delta+c\delta^2} \leq (1+\delta)^{1+\delta} \\
\end{gathered}
$$
Take log
$$
\begin{gathered}
    \delta+c\delta^2 \leq (1+\delta)\ln (1+\delta) \\
    c \leq \frac{(1+\delta)\ln (1+\delta) - \delta}{\delta^2} \\
\end{gathered}
$$

Thereby,

$$\mathbb P \{|S_N-\mu| \geq \delta\mu\} \leq 2e^{-c\mu\delta^2}$$

for all $c \leq [(1+\delta)\ln (1+\delta) - \delta]/\delta^2$.

\section{Exercise 2.3.8 (2 cc)}
(Normal approximation to Poisson)

Let $X \sim Pois(\lambda)$. Show that, as $\lambda \to \infty$, we have

$$\frac{X-\lambda}{\sqrt \lambda} \to \mathcal N(0, 1) \text{ in distribution.}$$

\subsection{Solution}

Rewrite the RV $X$ as a sum $Z_1 + \dots + Z_N$ where $Z_i \sim Pois(\lambda/N)$ and apply Lindeberg-Levy central limit theorem

$$\frac{X - \mathbb E X}{\sqrt{Var(X)}} = \frac{X - N \mathbb E Z_i}{\sqrt{N Var(Z_i)}} = \frac{X-\lambda}{\sqrt \lambda} \to \mathcal N(0, 1) \text{ as } N \to \infty$$

\section{Exercise 2.4.2 (1 cc)}

(Bounding the degrees of sparse graphs)

Consider a random graph $G \sim G(n,p)$ with expected degrees $d = O(\log n)$. Show that with high probability (say, 0.9), all vertices of $G$ have degrees $O(\log n)$.

\subsection{Solution}

Since $d_i > 0$ and $n > 0$, then for any $i$ we have $d_i = O(\log n) \equiv \exists C_i: d_i \leq C_i \log n$ and we need to prove the following statement
$$\mathbb P \{ \forall i : d_i \leq C_i \log n \} \geq 0.9$$

where $C_i$ is a constant. We can rewrite that probability as

$$\mathbb P \{ \exists i : d_i \geq C_i \log n \} \leq \sum \mathbb P \{ d_i \geq C_i \log n \} \leq 0.1$$

Apply Chernoff’s inequality for upper bound (since $d \leq C_i \log n$)

$$ \sum \mathbb P \{ d_i \geq C_i \log n \} \leq n e^{-d} \left(\frac{e d}{C_i \log n}\right)^{C_i \log n} \leq 0.1$$

Note that there exists $C_d$ such that $d \leq C_d \log n$ and then

$$n e^{-C_d \log n} \left(\frac{e C_d}{C_i}\right)^{C_i \log n} = n^{C_i + C_d + 1} \left(\frac{C_d}{C_i}\right)^{C_i \log n} \leq 0.1$$

Hence, the statement holds for any $C_i$ such that

$$C_i \geq \frac{C_d \log n + \log n + \log 10}{\log n } \left[W \left(\frac{C_d \log n + \log n + \log 10}{C_d e \log n}\right)\right]^{-1}$$

where $W(\cdot)$ is the product log function.

\section{Exercise 2.4.3 (2 cc)}

(Bounding the degrees of very sparse graphs)

Consider a random graph $G \sim G(n, p)$ with expected degrees $d = O(1)$. Show that with high probability (say, 0.9), all vertices of $G$ have degrees

$$O\left(\frac{\log n}{\log \log n}\right)$$

\subsection{Solution}

Apply the same approach as in the previous exercise and obtain

$$ \sum \mathbb P \left\{ d_i \geq C_i \frac{\log n}{\log \log n} \right\} \leq n e^{-C_d} \left(\frac{e C_d \log \log n}{C_i \log n}\right)^{(C_i \log n) / (\log \log n)} \leq 0.1$$

Hence, the statement holds for any $C_i$ such that

$$C_i \geq \frac{(\log n - C_d + \log 10)\log \log n}{\log n} \left[W\left( \frac{\log n - C_d + \log 10}{C_d e} \right)\right]^{-1}$$

where $W(\cdot)$ is the product log function.

\end{document}